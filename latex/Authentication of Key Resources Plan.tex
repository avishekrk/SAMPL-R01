\documentclass[11pt]{article}
\usepackage[top=0.5in, bottom=0.5in, left=0.5in, right=0.5in]{geometry}
\usepackage{helvet}
\usepackage{url} % hypderref?
\usepackage{graphicx}
\graphicspath{{figures/}} % The figures are in a figures/ subdirectory.
\renewcommand{\familydefault}{\sfdefault}
\pagestyle{empty}
%\pagestyle{plain}

\usepackage{setspace}

\usepackage{amsfonts}
\usepackage{amsmath}

\usepackage{sidecap}
\usepackage[abs]{overpic}
\usepackage{wrapfig}

%\usepackage[round,authoryear]{natbib}
\usepackage{cite}
%\setlength{\bibsep}{0.00in}

\usepackage{hyperref}
\hypersetup{colorlinks=true, urlcolor=black, citecolor=black, linkcolor=black}

\newcommand{\doi}[1]{\href{http://dx.doi.org/#1}{doi:#1}}

\newcommand{\ac}[1]{{\sc \lowercase{#1}}}

\renewcommand{\baselinestretch}{.93}
%\renewcommand{\baselinestretch}{.90}
\usepackage{wrapfig} 

\usepackage{bibspacing}
\setlength{\bibspacing}{\baselineskip}


\graphicspath{{figs/}}

\makeatletter

\newcommand{\captionfonts}{\footnotesize}

\makeatletter  % Allow the use of @ in command names
\long\def\@makecaption#1#2{%
  \vskip\abovecaptionskip
  \sbox\@tempboxa{{\captionfonts #1: #2}}%
  \ifdim \wd\@tempboxa >\hsize
    {\captionfonts #1. #2\par}
  \else
    \hbox to\hsize{\hfil\box\@tempboxa\hfil}%
  \fi
  \vskip\belowcaptionskip}      
\makeatother

\renewcommand{\figurename}{Fig.}

% Page numbering.
%\pagestyle{plain}
%\pagenumbering{arabic}

\setlength{\abovecaptionskip}{-5pt}

\makeatother

\renewcommand{\refname}{Bibliography and References Cited}

\setlength{\parindent}{0pt} % Don't indent first line
\setlength{\parskip}{1ex plus 0.5ex minus 0.2ex} % Add some space between paragraphs

\begin{document}

%======================================

%%%%%%%%%%%%%%%%%%%%%%%%%%%%%%%%%%%%%%%%%%%%%%%%%%%%%%%%%%%%%%%%%%%%%%%%%%%
% AUTHENTICATION OF KEY RESOURCES
%%%%%%%%%%%%%%%%%%%%%%%%%%%%%%%%%%%%%%%%%%%%%%%%%%%%%%%%%%%%%%%%%%%%%%%%%%%

%Authentication of Key Biological and/or Chemical Resources: Briefly describe methods to ensure the
%identity and validity of key biological and/or chemical resources used in the proposed studies.
%? Key biological and/or chemical resources may or may not be generated with NIH funds and: � May differ from laboratory to laboratory or over time;
%� May have qualities and/or qualifications that could influence the research data; and
%� Are integral to the proposed research.
%These include, but are not limited to, cell lines, specialty chemicals, antibodies, and other biologics.
%? Standard laboratory reagents that are not expected to vary do not need to be included in the plan. Examples are buffers and other common biologicals or chemicals.

\begin{centering}
{\bf AUTHENTICATION OF KEY RESOURCES PLAN}

\end{centering}

{\bf BIOLOGICAL RESOURCES}

{\bf Plasmid constructs.}
The sequence of engineered plasmid constructs of model proteins received or generated will be authenticated by antibiotic resistance marker and DNA sequencing of inserts in the cloning sites against canonical sequences in UniProt.

{\bf Bacterial cell lines} for expression of recombinant proteins and for molecular biology will be authenticated by their antibiotics profile and their genotype. 

{\bf CHEMICAL RESOURCES}

{\bf Small molecules (Chodera lab).}
Small molecules will be obtained from commercial sources.
These compounds will be characterized by HPLC-MS and $^1$H-NMR to verify their identity and purity as appropriate.
NMR spectra will be provided as supplementary material for reference.

{\bf Small molecules (Gibb lab).}
The purity of purchased speciality chemicals and substrates will be verified primarily by 1H NMR and chromatography. Log book notes for individual procedures will include the batch number and stated purity (and if needed, the determined purity) of each reagent utilized. 

{\bf Small molecules (Isaacs lab).}
Work performed in the Isaacs laboratory will utilize CB[n]-type container compounds that were previously reported along with drug molecules as guests that are commercially available. We will use the published literature procedures to resynthesize the needed container compounds and use spectroscopic methods to verify their identity and purity. Similarly, drug molecules as guest will be verified for identity and purity by analysis of their NMR spectra which will be deposited in the supporting information of the corresponding publications. Samples of all compounds synthesized under this grant will be retained in the Isaacs laboratory at the University of Maryland and will be made available to researchers upon request. Such requests will need to be made to the University of Maryland Office of Technology Commercialization which will execute a standard material transfer agreement with the requestor.

{\bf Recombinantly expressed proteins.}
Recombinant proteins will either be produced in-house or obtained from commercial sources.
The molecular weight, concentration, and purity of purified His-tagged recombinantly expressed proteins will be verified using a Caliper GXII microfluidic gel electrophoresis instrument.
ThermoFluor melts (thermal denaturation scans in the presence of Cypro Orange, a dye that changes fluorescence upon binding to unfolded proteins) performed using a Roche LC480 qPCR machine will be used to verify protein stability in our buffer systems.

{\bf Buffers (Chodera lab).}
Buffers used for various biophysical assays are produced in a reproducible fashion by a LabMinds Revo automated buffer maker, which automatically prepares buffers in a reproducible manner, adjusting pH and filtering automatically.

{\bf Buffers (Gibb lab).}
Buffers utilized in NMR and ITC titration experiments will be made in-house using ultra-pure water, the purity of which will be verified by on-the-spot conductance measurements.

{\bf Buffers (Overall).}
Complete details of all buffers (such as final pH, exact composition of buffer components by mass, manufacturer and lot numbers of all components) are stored online and will be made available as supplementary material.

%======================================

%\setlength{\bibsep}{0.000in}

%\bibliographystyle{gec_nih}
%\bibliographystyle{ieeetr}
%\bibliographystyle{myrefstyle}
%\bibliography{chodera-research}


\end{document}






